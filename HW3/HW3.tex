\documentclass[11pt]{article}

\usepackage{amsfonts}

\usepackage{fancyhdr}
\usepackage{amsmath}
\usepackage{amsthm}
\usepackage{amssymb}
\usepackage{amsrefs}
\usepackage{ulem}
\usepackage[dvips]{graphicx}
\usepackage{color}
\usepackage{bm}
\usepackage{cancel}

\setlength{\headheight}{26pt}
\setlength{\oddsidemargin}{0in}
\setlength{\textwidth}{6.5in}
\setlength{\textheight}{8.5in}

\topmargin 0pt
%Forrest Shortcuts
\newtheorem{defn}{Definition}
\newtheorem{thm}{Theorem}
\newtheorem{lemma}{Lemma}
\newtheorem{pf}{Proof}
\newtheorem{sol}{Solution}
\newcommand{\R}{{\ensuremath{\mathbb R}}}
\newcommand{\J}{{\ensuremath{\mathbb J}}}
\newcommand{\Z}{{\mathbb Z}}
\newcommand{\N}{{\mathbb N}}
\newcommand{\T}{{\mathbb T}}
\newcommand{\Q}{{\mathbb Q}}
\newcommand{\st}{{\text{\ s.t.\ }}}
\newcommand{\rto}{\hookrightarrow}
\newcommand{\rtto}{\hookrightarrow\rightarrow}
\newcommand{\tto}{\to\to}
\newcommand{\C}{{\mathbb C}}
\newcommand{\ep}{\epsilon}
%CJ shortcuts
\newcommand{\thin}{\thinspace}
\newcommand{\beps}{\boldsymbol{\epsilon}}
\newcommand{\bwoc}{by way of contradiction}

%Munkres formatting?
%\renewcommand{\theenumii}{\alph{enumi}}
\renewcommand{\labelenumi}{\theenumi.}
\renewcommand{\theenumii}{\alph{enumii}}
\renewcommand{\labelenumii}{(\theenumii)}

\title{HW3}
\author{Guanlin Zhang}

\lhead{Dr Milind Phadnis
 \\BIOS 845} \chead{}
\rhead{Guanlin Zhang\\ Spring '18} \pagestyle{fancyplain}
%\maketitle

\begin{document}

Question $\# 1$:
\begin{sol}
	For part $(A)$, \vskip 2mm
	we have the PH model as:
	\begin{align*}
		h(t, \text{age}, \text{sex}, \text{age}*\text{sex}) = h_0(t)\exp\Big(\beta_1 \text{age} + \beta_2\text{sex} + \beta_3\text{sex}*\text{age}\Big)
	\end{align*}	
	The following SAS code fit the data into this model:
	\begin{center}
		\includegraphics[width = 10cm]{Q1A01.jpg}
	\end{center}
	The output is:
	\begin{center}
		\includegraphics[width = 12cm]{Q1A02.jpg}
	\end{center}
	We see that the interaction between age and sex is only marginally significant, and the effect of sex is also marginally significant.\vskip 2mm
	To find the most parsimonious model, we tried forward, backward and stepwise methods to select explanatory variables:
	\begin{center}
		\includegraphics[width = 10cm]{Q1A03.jpg}
	\end{center}
	All three methods got the same result, that only age is kept in the model.
	\begin{center}
		\includegraphics[width = 12cm]{Q1A04.jpg}
	\end{center}
	So the most parsimonious model is:
	\begin{align*}
		h(t, \text{age}) = h_0(t)\exp\Big(\beta_1 \cdot  \text{age}\Big)
	\end{align*}
	From the output we conclude that for every one year increase in age, the risk of death increase about $4.5\%$ at that time.\vskip 2mm
	Part $(B)$:\vskip 2mm
	Since we are using the full model, we need to consider the interaction. Thus the hazard ratio for sex, adjusting for age will be:
	\begin{align*}
		\text{HR} &= \frac{h_0(t)\exp\Big(\beta_1\cdot \text{age} + \beta_2\cdot 1 + \beta_3\cdot 1\cdot \text{age}\Big)}{h_0(t)\exp\Big(\beta_1\cdot\text{age} + \beta_2\cdot 0 + \beta_3\cdot 0\cdot \text{age}\Big)} = \exp\Big(\beta_2 + \beta_3\cdot \text{age}\Big)\\
		&= \exp\Big(1.55289 -0.02040\times \text{age}\Big)
	\end{align*}
	For age at $50$, 
	\begin{align*}
		\text{HR} &= \exp\Big(1.55289 -0.02040\times 50\Big) = 1.7038
	\end{align*}
	For age at $60$, 
	\begin{align*}
		\text{HR} &= \exp\Big(1.55289 -0.02040\times 60\Big) = 1.3894
	\end{align*}
	For age at $65$, 
	\begin{align*}
		\text{HR} &= \exp\Big(1.55289 -0.02040\times 65\Big) = 1.2547
	\end{align*}
	For age at $70$,
	\begin{align*}
		\text{HR} &= \exp\Big(1.55289 -0.02040\times 70\Big) = 1.1330
	\end{align*}
	For age at $80$,
	\begin{align*}
		\text{HR} &= \exp\Big(1.55289 -0.02040\times 80\Big) = 0.9239
	\end{align*}
	Now for male, the hazard ratio for a $10$ year increase is:
	\begin{align*}
		\text{HR} &= \frac{h_0(t)\exp\Big(\beta_1(\text{age} + 10) + \beta_2\cdot 1 + \beta_3\cdot 1 \cdot (\text{age + 10})\Big)}{h_0(t)\exp\Big(\beta_1\cdot \text{age} + \beta_2\cdot 1 + \beta_3\cdot 1\cdot \text{age}\Big)}\\
		&= \exp\Big(10\beta_1 + 10\beta_3\Big) = \exp\Big(10(\beta_1 + \beta_3)\Big)\\
		&= \exp\Big(10(0.05175-0.02040)\Big)\\
		&= 1.3682
	\end{align*}
	For female, the hazard ratio for a $10$ year increase in age is:
	\begin{align*}
		\text{HR} &= \frac{h_0(t)\exp\Big(\beta_1(\text{age} + 10) + \beta_2\cdot 0 + \beta_3\cdot 0 \cdot (\text{age} + 10)\Big)}{h_0(t)\exp\Big(\beta_1(\text{age}) + \beta_2\cdot 0 + \beta_3\cdot 0 \cdot \text{age}\Big)}\\
		&= \exp\Big(10\beta_1\Big) = \exp\Big(10\times 0.05175\Big)\\
		&= 1.6778
	\end{align*}
	\vskip 2mm
	For part $(C)$:\vskip 2mm
	The estimate for survival function after adjusting for covariates is given by:
	\begin{align*}
		\hat{S}(t) = \Big[\hat{S}_0(t)\Big]^{\exp\Big(\hat{\beta}_1\text{age} + \hat{\beta}_2\text{sex} + \hat{\beta}_3\text{age}*\text{sex}\Big)}
	\end{align*}
	where
	\begin{align*}
		\hat{S}_0(t) &= exp\Big[-\hat{H}_0(t)\Big]
	\end{align*}
	and
	\begin{align*}
		\hat{H}_0(t) &= \sum_{t_i < t}\frac{1}{\sum_{j \in R(t_i)}\exp\Big(\hat{\beta}_1\text{age}_j + \hat{\beta}_2\text{sex}_j + \hat{\beta}_3\text{age}_j*\text{sex}_j\Big)} \text{ (if there is no tie)}\\
		\hat{H}_0(t) &= \sum_{t_i < t}\frac{d_i}{\sum_{j \in R(t_i)}\exp\Big(\hat{\beta}_1\text{age}_j + \hat{\beta}_2\text{sex}_j + \hat{\beta}_3\text{age}_j*\text{sex}_j\Big)} \text{ (if there are ties)}
	\end{align*}
	The code to realize the above results is:
	\begin{center}
		\includegraphics[width = 12cm]{Q1C01.jpg}
	\end{center}
	and the plot of the survival curves for male and female at age 65 is:
	\begin{center}
		\includegraphics[width = 12cm]{Q1C02.jpg}
	\end{center}
	To estimate the median survival times, recall that by definition:
	\begin{align*}
		x_{0.5} &= \inf\{t: S(t) \leq 0.5\} \text{ with estimate } \hat{x}_{0.5} = \inf\{t: \hat{S}(t) \leq 0.5\}
	\end{align*}
	We print out the estimate for survival function at age 65 for both genders:
	\begin{center}
		female: \includegraphics[width = 5cm]{Q1C03.jpg} male: \includegraphics[width = 5cm]{Q1C04.jpg}
	\end{center}
	So the estimated median survival times are:
	\begin{align*}
		\text{median}_{\text{female}} &= 3361, \hskip 1cm \text{median}_{\text{male}} = 2217
	\end{align*}
	For part $D$:\vskip 2mm
	We use $\text{AGE}_\text{CAT}$ in place of $\text{AGE}$, and we also consider the interaction between $\text{AGE}\_\text{CAT}$ and $\text{SEX}$. Since there are $4$ categories in $\text{AGE}\_\text{CAT}$, in order for SAS to be able to handle it, we create our own dummy variables (thus having more control over the class statement).\vskip 2mm
	We define:
	\begin{align*}
		&\ \text{CAT}1 = \left\{\begin{array}{ll} 1& \text{ if } \text{AGE}\_\text{CAT} = 1\\ 0& \text{otherwise} \end{array}\right.\hskip 2cm \text{CAT}2 = \left\{\begin{array}{ll} 1& \text{ if } \text{AGE}\_\text{CAT} = 2\\ 0& \text{otherwise} \end{array}\right.\\
		&\ \text{CAT}3 = \left\{\begin{array}{ll} 1& \text{ if } \text{AGE}\_\text{CAT} = 3\\ 0& \text{otherwise} \end{array}\right.
	\end{align*} 
	Hence we are using the last age category($4 = 78+$) as a reference category and our model is:
	\begin{align*}
		h(t) &= h_0(t)\cdot \exp\Big(\beta_1\cdot \text{CAT}1 + \beta_2\cdot \text{CAT}2 + \beta_3\cdot \text{CAT}3 + \beta_4\cdot \text{SEX} \\
		&\ \hskip 1cm + \beta_5 \cdot \text{CAT}1\cdot \text{SEX} + \beta_6 \cdot \text{CAT}2\cdot \text{SEX} + \beta_7\cdot \text{CAT}3\cdot \text{SEX}\Big)
	\end{align*}
	The following SAS code fit the above model to our data:
	\begin{center}
		\includegraphics[width = 14cm]{Q1D01.jpg}
	\end{center}
	We would like to mention that, if we do not want to manually create the dummy variables as above, equivalently we could let SAS to do it for us with the class statement. The following code will produce the exact same output:
	\begin{center}
		\includegraphics[width = 8cm]{Q1D01_a.jpg}
	\end{center}
	and we can check how SAS create the dummy variable by default:
	\begin{center}
		\includegraphics[width = 5cm]{Q1D01_b.jpg}
	\end{center}
	which is the same as our definition above.\vskip 2mm
	The output is:
	\begin{center}
		\includegraphics[width = 12cm]{Q1D02.jpg}
	\end{center}
	All main effect terms are significant, and among the interaction terms, only $\text{cat}1\ast \text{SEX}$ is marginally significant(p value $0.05$) and the others are also significant.\vskip 2mm
	If we add option $\text{selection}= \ldots $  after the model statement, it will only keep the three main effect $\text{cat}1-3$ and get rid of $\text{sex}$(regardless if we use stepwise, forward or backward). Since all covariates and interactions are significant, we decide to keep the full model, consider we are interested in the hazard ratio of SEX later.\vskip 2mm
	Now for age group $[24, 60)$, the hazard ratio for SEX is:
	\begin{align*}
		\text{HR} &= \frac{h_0(t)\exp\Big(\beta_1\cdot 1 + \beta_4\cdot 1 + \beta_5\cdot 1\cdot 1\Big)}{h_0(t)\exp\Big(\beta_1\cdot 1 + \beta_4\cdot 0 + \beta_5\cdot 1 \cdot 0\Big)}\\
		&= \exp\Big(\beta_4 + \beta_5\Big) = \exp\Big(-0.54991 + 0.83654\Big)\\
		&= 1.332
	\end{align*}
	For age group $[60, 69)$, the hazard ratio for SEX is:
	\begin{align*}
		\text{HR} &= \frac{h_0(t)\exp\Big(\beta_2\cdot 1 + \beta_4\cdot 1 + \beta_6\cdot 1\cdot 1\Big)}{h_0(t)\exp\Big(\beta_2\cdot 1 + \beta_4\cdot 0 + \beta_6\cdot 1\cdot 0\Big)}\\
		&= \exp\Big(\beta_4 + \beta_6\Big) = \exp\Big(-0.54991 + 0.75854\Big)\\
		&= 1.232
	\end{align*}
	For age group $[69, 78)$, the hazard ratio for SEX is:
	\begin{align*}
		\text{HR} &= \exp\Big(\beta_4 + \beta_7\Big) = \exp\Big(-0.54991 + 0.93861\Big)\\
		&= 1.475
	\end{align*}
	For age group $78+$, the hazard ratio for SEX is:
	\begin{align*}
		\text{HR} &= \frac{h_0(t)\cdot \exp\Big(\beta_4\cdot 1\Big)}{h_0(t)\cdot \exp\Big(\beta_4\cdot 0\Big)} = \exp\Big(\beta_4\Big) = \exp \Big(-0.54991\Big) = 0.5770
	\end{align*}
	\vskip 2mm
%	We fit AGE$\_$CAT, SEX, and the interaction of AGE$\_$CAT and SEX into the cox regression model. The code is:
%	\begin{center}
%		\includegraphics[width = 10cm]{Q1D01_a.jpg}
%	\end{center}
%	The output is:
%%	\begin{center}
%%		\includegraphics[width = 4cm]{Q1D02.jpg}
%%	\end{center}
%	\begin{center}
%		\includegraphics[width = 14cm]{Q1D03_a.jpg}
%	\end{center}
%	All covariates are significant.\vskip 2mm
%	Keep in mind that our PH model here is:
%	\begin{align*}
%		h(t) = h_0(t)\cdot \exp\Big(\beta_1\text{age}\_\text{cat} + \beta_2\text{sex} + \beta_3\text{agecatsex}\Big)
%	\end{align*}
%	For age category $1$: $[24, 60)$, we have:
%	\begin{align*}
%		&\ \text{for male:}\\
%		&\ h(t) = h_0(t)\cdot \exp\Big(\beta_1 + \beta_2 + \beta_3\Big)\\
%		&\ \text{for female:}\\
%		&\ h(t) = h_0(t)\cdot \exp\Big(\beta_1 + 0 + 1\cdot 0\Big)
%	\end{align*}
%	So the hazard ratio for age group $1$ is:
%	\begin{align*}
%		HR_1 &= \exp\Big(\beta_2 + \beta_3\Big) = \exp\Big(0.81743 - 0.28117\Big) = 1.7096
%	\end{align*}
%	We conclude that for people in the age range $[24, 60)$, male's death risk is about $1.71$ times of female.\vskip 2mm
%	For age category $2$: $[60, 69)$, we have:
%	\begin{align*}
%		&\ \text{for male:}\\
%		&\ h(t) = h_0(t)\cdot \exp\Big(2\beta_1 + \beta_2 + 2\beta_3\Big)\\
%		&\ \text{for female:}\\
%		&\ h(t) = h_0(t)\cdot \exp\Big(2\beta_1 + 0 + 2\cdot 0\Big)
%	\end{align*}
%	So the hazard ratio for age group $2$ is:
%	\begin{align*}
%		HR_2 &= \exp\Big(\beta_2 + 2\beta_3\Big) = \exp\Big(0.81743 -2 \times 0.28117\Big) = 1.290578
%	\end{align*}
%	we conclude that for people in the age range $[60, 69)$, male's death risk is about $1.30$ times of female.\vskip 2mm
%	For age category $3$: $[69, 78)$, we have:
%	\begin{align*}
%		&\ \text{for male:}\\
%		&\ h(t) = h_0(t)\cdot \exp\Big(3\beta_1 + \beta_2 + 3\beta_3\Big)\\
%		&\ \text{for female:}\\
%		&\ h(t) = h_0(t)\cdot \exp\Big(3\beta_1 + 0 + 3\cdot 0\Big)
%	\end{align*}
%	So the hazard ratio for age group $2$ is:
%	\begin{align*}
%		HR_3 &= \exp\Big(\beta_2 + 3\beta_3\Big) = \exp\Big(0.81743 -3 \times 0.28117\Big) = 0.9742571
%	\end{align*}
%	we conclude that for people in the age range $[69, 78)$, male's death risk is about $97\%$ of female.\vskip 2mm
%	For age category $4$: $78+$, we have:
%	\begin{align*}
%		&\ \text{for male:}\\
%		&\ h(t) = h_0(t)\cdot \exp\Big(4\beta_1 + \beta_2 + 4\beta_3\Big)\\
%		&\ \text{for female:}\\
%		&\ h(t) = h_0(t)\cdot \exp\Big(4\beta_1 + 0 + 4\cdot 0\Big)
%	\end{align*}
%	So the hazard ratio for age group $2$ is:
%	\begin{align*}
%		HR_4 &= \exp\Big(\beta_2 + 4\beta_3\Big) = \exp\Big(0.81743 -4 \times 0.28117\Big) = 0.7354667
%	\end{align*}
%	we conclude that for people in the age range $78+$, male's death risk is about $73.5\%$ of female.\vskip 2mm
	For part $E$:\vskip 2mm
	I would prefer to report the results from model in part-D. The model in part-A says that only effect of age is significant, while the sex and interaction between sex and age are only marginally significant.  The model in part-D conclude that age, sex and their interactions are all significant. From the plot in part-C we see that apparently female tends to have a longer survival time than male, given age is $65$. So adjusting for age, we are expecting to see a significant difference on survival time between different sex. This is better supported by the model in part-D.\vskip 2mm
	Since the model we choose is:
	\begin{align*}
		h(t) &= h_0(t)\cdot \exp\Big(\beta_1\cdot \text{CAT}1 + \beta_2\cdot \text{CAT}2 + \beta_3\cdot \text{CAT}3 + \beta_4\cdot \text{SEX} \\
		&\ \hskip 1cm + \beta_5 \cdot \text{CAT}1\cdot \text{SEX} + \beta_6 \cdot \text{CAT}2\cdot \text{SEX} + \beta_7\cdot \text{CAT}3\cdot \text{SEX}\Big)
	\end{align*}
	We run the following SAS code to acquire partial maximum likelihood estimate:
	\begin{center}
		\includegraphics[width = 14cm]{Q1E01.jpg}
	\end{center}
	\begin{center}
		\includegraphics[width = 16cm]{Q1E02.jpg}
	\end{center}
	we can then compute the $95\%$ confidence limits for the regression coefficients:
	\begin{align*}
		&\ \text{CI}_{\beta_1}:  \Big(\log (0.084), \log (0.226)\Big) = \Big(-2.4769, -1.4872\Big)\\
		&\ \text{CI}_{\beta_2}:  \Big(\log (0.142), \log (0.374)\Big) = \Big(-1.9519,-0.9835 \Big)\\
		&\ \text{CI}_{\beta_3}:  \Big(\log (0.234), \log (0.586)\Big) = \Big(-1.4524, -0.5344\Big)\\
		&\ \text{CI}_{\beta_4}:  \Big(\log (0.376), \log (0.886)\Big) = \Big(-0.9782, -0.1210\Big)\\
		&\ \text{CI}_{\beta_5}:  \Big(\log (0.996), \log (5.352)\Big) = \Big(-0.0040, 1.6775\Big)\\
		&\ \text{CI}_{\beta_6}:  \Big(\log (1.059), \log (4.304)\Big) = \Big(0.05733, 1.4595\Big)\\
		&\ \text{CI}_{\beta_7}:  \Big(\log (1.364), \log (4.791)\Big) = \Big(0.3104, 1.5667\Big)
	\end{align*}
	We also have the global test for $H_0: \beta_1 = \beta_2 = \beta_3 = \beta_4= \beta_5 = \beta_6 = \beta_7 = 0$
	\begin{center}
		\includegraphics[width = 6cm]{Q1E03.jpg}
	\end{center}
	For part $(F)$:\vskip 2mm
	1) We plot Schoenfeld Residual for age versus the length of follow up. Since there is no obvious pattern, we find no violation of the PH model assumption for age.\vskip 2mm
	2) We plot log-log survival functions versus log of length of follow up for both male and female. Since the plots are parallel, we find no violation of the PH model assumption for sex.\vskip 2mm
	The code for $1)$ and $2)$ is:
	\begin{center}
		\includegraphics[width = 8cm]{Q1F01.jpg}
	\end{center}
	The plot for $1)$ is:
	\begin{center}
		\includegraphics[width = 12cm]{Q1F02.jpg}
	\end{center}
	The plot for $2)$ is:
	\begin{center}
		\includegraphics[width = 12cm]{Q1F03.jpg}
	\end{center}
	We can further check the PH model assumption with:\vskip 2mm
	3) We plot the observed and expected survival function for both male and female to check on sex. \vskip 2mm
	The plot is:
	\begin{center}
		\includegraphics[width = 12cm]{Q1F05.jpg}
	\end{center}
	generated by the following code:
	\begin{center}
		\includegraphics[width = 10cm]{Q1F04.jpg}
	\end{center}
	For both male and female, the observed and expected survival functions are pretty close, so we find no violation of the PH model assumption for sex.\vskip 2mm
	4) We consider the interaction with time for both age and sex. The code is:
	\begin{center}
		\includegraphics[width = 10cm]{Q1F06.jpg}
	\end{center}
	The output is:
	\begin{center}
		\includegraphics[width = 12cm]{Q1F07.jpg}
	\end{center}
	For both age and sex, the interactions with time are not significant, hence we find no violation of PH assumption for both age and sex.\vskip 2mm
	To conclude, 1)-4) all support the PH model assumption we have made for age and sex.\vskip 2mm
	For part $(G)$:\vskip 2mm
	To test the goodness of fit, we first look at the log-survival plot of coxsnell residuals given by the following code:
	\begin{center}
		\includegraphics[width = 10cm]{Q1G01.jpg}
	\end{center}
	The plot is:
	\begin{center}
		\includegraphics[width = 12cm]{Q1G02.jpg}
	\end{center}
	It is not deviating much from the straightline, so the coxsnell residual test tells us that the PH model from part $F$ we fitted is adequate.\vskip 2mm
	To see whether the functional form of age is appropriate (we are assuming linear form here, namely, $\beta_1 \ast \text{age}$), we consider using martingale residuals. We plot martingale residuals against age, along with a LOESS smoothing function. We also use ASSESS keyword in proc phreg to plot the observed cumulative martingale residual for age together with $20$ simulated realizations, as well as a 4-panel display (observed cumulative martingale residual process for age together with the first 8 siulated realizations).\vskip 2mm
	The code is:
	 \begin{center}
		\includegraphics[width = 9cm]{Q1G03.jpg}
	\end{center}
	The martingale residual against age plot is:
	 \begin{center}
		\includegraphics[width = 10cm]{Q1G04.jpg}
	\end{center}
	There does not seem to be an obvious pattern, so we conclude the current functional form for age is appropriate.\vskip 2mm
	The cumulative martingale residual process for age is:
	 \begin{center}
		\includegraphics[width = 7cm]{Q1G05.jpg}\includegraphics[width = 7cm]{Q1G06.jpg}
	\end{center}
	As we can see the observed cumulative martingale residual process is within the spectrum of simulated ones, again supporting the current functional form of age.\vskip 2mm
	We next identify the outliers by checking deviance residuals:\vskip 2mm
	We sketch the deiviance residual versus age to see the adequacy of fit for age, and use proc univariate to identity extreme observations(outliers).\vskip 2mm
	The SAS code and output is:
	\begin{center}
		\includegraphics[width = 8cm]{Q1G07.jpg}
	\end{center}
	\begin{center}
		\includegraphics[width = 12cm]{Q1G08.jpg}
	\end{center}
	\begin{center}
		\includegraphics[width = 4cm]{Q1G09.jpg}
	\end{center}
	The residual plot seem to show that there are more observations with positive values (shorter survival times than expected), but there is no obvious pattern to indicate any trend. We still think that age is adequately fit.\vskip 2mm
	The proc univariate output finds outliers for us as shown above.\vskip 2mm
	Finally we use the DFBETA keyword under the output statement of proc phreg to identify influential observations.Keep in mind that DFBETA compute $\hat{\beta} - \hat{\beta}_{(j)}$ where $\hat{\beta}_{(j)}$ is the regression coefficient estimate with observation $j$ omitted. A positive value means higher hazard rate estimate with observation $j$ involved compared to without, hence observation $j$ refers to an individual who has moderate values of the covariate corresponding to $\beta$ but experience event earlier than expected. On the other hand, a negative value means lower hazard rate estimate with observation $j$ involved compared to without, hence observation $j$ refers to an individual who has small values of the covariate but experience the event at a time longer than expected.\vskip 2mm
	The SAS code for computing the $\hat{\beta} - \hat{\beta}_{(j)}$ corresponding to each of the covariate age and sex is:
	\begin{center}
		\includegraphics[width = 10cm]{Q1G10.jpg}
	\end{center}
	we also plot the difference in the parameter for age versus observation number as:
	\begin{center}
		\includegraphics[width = 12cm]{Q1G11.jpg}
	\end{center}
	The list of $\hat{\beta} - \hat{\beta}_{(j)}$ for all $481$ observations are too long, so we did not print it out here in our summary. But we printed out it in SAS and find the following influential obervations(we are focusing on age because sex is not significnat in our model):\vskip 2mm
	Observation $43, 60, 160, 260$ and $465$ are with relatively large negative values.\vskip 2mm
	We actually did not find any influential observations with relatively large positive differences on $\hat{\beta} - \hat{\beta}_{(j)}$ for age, at least not on the same level of those negative differences.\vskip 2mm
	Thus completed the solution for Question $\# 1$.
\end{sol}

Question $\# 2$:
\begin{sol}
	For part $(A)$, our PH model is:
	\begin{align*}
		h(t) = h_0(t)\cdot \exp\Big(\beta_1 \cdot \text{age} + \beta_2\cdot \text{drug}\Big)
	\end{align*}
	Although drug only takes values on either $50$ or $100$, we are treating it as a continuous covariate here. The SAS code is as following:
	\begin{center}
		\includegraphics[width = 8cm]{Q2A01.jpg}
	\end{center}
	and we got the following estimate:
	\begin{center}
		\includegraphics[width = 10cm]{Q2A02.jpg}
	\end{center}
	The effect of drug is {\bf NOT}significant at the $0.05$ significance level(p value $0.94$). If we ignore this fact that the drug as a continuous covariate is highly insignificant, we do have a point estimate that says $\hat{\beta}_2 = 0.0004632$, which leads to a hazard ratio adjusting for age as $\exp\Big(0.0004632\Big) \simeq 1.000$. This says that for each single unit increase in the drug dosage, the chance of cure(risk of event) is about the same as before the increase, namely, the drug is useless.\vskip 2mm
	For part $(B)$:\vskip 2mm
	Take the amount of drug remaining in the body by the end of study as a covariate, our model is:
	\begin{align*}
		h(t) = h_0(t)\exp\Big(\beta_1 \cdot \text{age} + \beta_2 \cdot \text{remaining drug}\Big)
	\end{align*}
	The following SAS code fit this model:
	\begin{center}
		\includegraphics[width = 8cm]{Q2B01.jpg}
	\end{center}
	and the output is:
	\begin{center}
		\includegraphics[width = 10cm]{Q2B02.jpg}
	\end{center}
	under the new model the effect of drug (remaining amount in the body) is significant at the $0.05$ level. The point estimate for the drug effect here is $\hat{\beta}_2 = 0.54873$ which leads to a hazard ratio adjusted for age as $\exp(0.54873) = 1.731$. It tells that for every unit increase of remaining drug dosage in the body by the end of study,  the chance of cure(risk of event) is about $1.731$ times of the chance without increase.\vskip 2mm
	For part $(C)$:\vskip 2mm
	Our model is:
	\begin{align*}
		h(t) = h_0(t)\exp\Big(\beta_1 \cdot \text{age} + \beta_2 \cdot \text{initial dosage}\cdot \exp(-0.35t)\Big)
	\end{align*}
	Thus we are looking at a model with time dependent covariate
	\vskip 2mm
	The code is:
	\begin{center}
		\includegraphics[width = 9cm]{Q2C01.jpg}
	\end{center}
	and the output is:
	\begin{center}
		\includegraphics[width = 10cm]{Q2C02.jpg}
	\end{center}
	To compute the hazard ratio between patient with an initial dosage level of $100$ mg and patient with an initial dosage level of $50$ mg at day $7$, adjusting for age (assuming two patients at same age), we have:
	\begin{align*}
		\text{HR} &= \frac{h_0(t)\exp\Big(\hat{\beta}_1\cdot \text{age} + \hat{\beta}_2\cdot 100\cdot \exp\Big(-0.35\times 7\Big)\Big)}{h_0(t)\exp\Big(\hat{\beta}_1\cdot \text{age} + \hat{\beta}_2\cdot 50\cdot \exp\Big(-0.35\times 7\Big)\Big)} = \exp\Big(\hat{\beta}_2\cdot 50\cdot \exp\Big(-0.35\times 7\Big)\Big)\\
		&= \exp\Big(0.17855\cdot 50\cdot \exp\Big(-0.35\times 7\Big)\Big)\\
		&= 2.1606
	\end{align*}
	This is to say, at $7$ days, the patient having initial dosage of $100$ mg has $2.1606$ times the chance of cure of patient having initial dosage of $50$mg, adjusting for age.\vskip 2mm
	Similarly, we can compute the hazard ratio at day $14$:
	\begin{align*}
		\text{HR} &= \exp\Big(\hat{\beta}_2\cdot 50\cdot \exp\Big(-0.35\times 14\Big)\Big)\\
		&= \exp\Big(0.17855\cdot 50\cdot \exp\Big(-0.35\times 14\Big)\Big)\\
		&= 1.0687
	\end{align*}
	We conclude that at $14$ days, the patient having initial dosage of $100$ mg has $1.0687$ times the chance of cure of patient having initial dosage of $50$ mg.
	\vskip 2mm
	Thus complete the answers for question $\# 2$.
\end{sol}

Question $\#3$:
\begin{sol}
	For part $(A)$:\vskip 2mm
	The extended Cox model is:
	\begin{align*}
		h(t) = h_0(t)\exp\Big(\beta_1 \cdot \text{RX} + \beta_2\cdot \text{LOGWBC} + \beta_3\cdot \text{SEX}\times\text{TIME}(t)\Big)
	\end{align*}	
	Here TIME is a new variable defined as:
	\begin{align*}
		\text{TIME}(t) = \left\{\begin{array}{ll} 1& \text{ if } t < 4 \\ 3& \text{ if } 4 \leq t < 8\\ 5& \text{ if }8 \leq t < 12\\ 7& \text{ if }12 \leq t < 16\\ 9& \text{ if }t \geq 16\end{array}\right.
	\end{align*}
	and then SEX$\times$TIME$(t)$ is a time dependent covariate as given by the question.\vskip 2mm
	For part $(B)$:\vskip 2mm
	Adjusted for RX and LOGWBC, the hazard ratio for the effect of SEX is:
	\begin{align*}
		\text{HR}_{\text{SEX}} &= \frac{h_0(t)\exp\Big(\beta_1\cdot\text{RX} + \beta_2\cdot\text{LOGWBC}+ \beta_3\cdot 1 \cdot \text{TIME}(t)\Big)}{h_0(t)\exp\Big(\beta_1\cdot\text{RX} + \beta_2\cdot\text{LOGWBC} + \beta_3\cdot 0 \cdot \text{TIME}(t)\Big)}\\
		&= \exp\Big(\beta_3\cdot \text{TIME}(t)\Big) = \left\{\begin{array}{ll} \exp\Big(\beta\Big)& \text{ if } t < 4\\ \exp\Big(3\beta\Big)& \text{ if } 4 \leq t < 8\\ \exp\Big(5\beta\Big) &\text{ if }8 \leq t < 12 \\ \exp\Big(7\beta)& \text{ if }12 \leq t < 16\\ \exp\Big(9\beta\Big)&  \text{ if }t \geq 16\end{array}\right.
	\end{align*}
	For part $(C)$:\vskip 2mm
	To fit the model developed from part $(A)$, we present the following SAS code:
	\begin{center}
		\includegraphics[width = 9cm]{Q3A01.jpg}
	\end{center}
	The output is:
	\begin{center}
		\includegraphics[width = 12cm]{Q3A02.jpg}
	\end{center}
	We have an estimate $\hat{\beta}_3 =-0.03549$. To estimate the hazard ratio for the effect of sex, using the formula we have developed in part B, we got:
	\begin{align*}
		\hat{\text{HR}}_{\text{SEX}}
		&= \exp\Big(\hat{\beta}_3\cdot \text{TIME}(t)\Big) = \left\{\begin{array}{ll} \exp\Big(\hat{\beta}\Big)& \text{ if } t < 4\\ \exp\Big(3\hat{\beta}\Big)& \text{ if } 4 \leq t < 8\\ \exp\Big(5\hat{\beta}\Big) &\text{ if }8 \leq t < 12 \\ \exp\Big(7\hat{\beta}\Big)& \text{ if }12 \leq t < 16\\ \exp\Big(9\hat{\beta}\Big)&  \text{ if }t \geq 16\end{array}\right. = \left\{\begin{array}{ll} 0.9651 & \text{ if } t < 4\\ 0.8990& \text{ if } 4 \leq t < 8\\  0.8374&\text{ if }8 \leq t < 12 \\ 0.7800& \text{ if }12 \leq t < 16\\ 0.7266&  \text{ if }t \geq 16\end{array}\right.
	\end{align*}
	For part $(D)$:\vskip 2mm
	To fit the cox model stratified on sex, we present the following SAS code:
	\begin{center}
		\includegraphics[width = 9cm]{Q3D01.jpg}
	\end{center}
	The output for female is:
	\begin{center}
		\includegraphics[width = 9cm]{Q3D02.jpg}
	\end{center}
	The output for male is:
	\begin{center}
		\includegraphics[width = 9cm]{Q3D03.jpg}
	\end{center}
	For part $(E)$:\vskip 2mm
	The hazard ratio of RX from part $C$ is:
	\begin{align*}	
		\hat{\text{HR}}_{C, \text{RX}} = 3.332
	\end{align*}
	while the hazard ratio of RX from part $D$, depends on the gender, is:
	\begin{align*}
		\hat{\text{HR}}_{D, \text{RX}, \text{male}} &= 6.418\\
		\hat{\text{HR}}_{D, \text{RX}, \text{female}} &= 1.306
	\end{align*}
	It appears that we got very different hazard ratios between part $C$ and part $D$. However we would not call out either one of them as being more appropriate. The reason is that part $C$ and part $D$ are taking different point of views. Part $C$ is modeling the effect of sex as a time dependent covariate, so when we are looking at the hazard ratio for RX, we are adjusting for the effect of sex as well. However for part $D$, the sex is not a covariate in the model, instead we stratified on the sex and only consider the hazard ratio of RX on the same gender. So the hazard ratios between C and D are not comparable.\vskip 2mm
	Thus complete the answers for question $\# 3$.
\end{sol}

Question $\#4$.
\begin{sol}
	For part $(A)$:\vskip 2mm
	For Model $1$, we have all $5$ covariates in the model and the cox regression model is:
	\begin{align*}
		h(t) = h_0(t)\cdot \exp\Big(\beta_1\cdot\text{PLTLTS} + \beta_2\cdot \text{AGE} + 
			\beta_3\cdot \text{SEX} + \beta_4\cdot \text{PLTAGE} + \beta_5\cdot \text{PLTSEX}\Big)
	\end{align*}
	For model $2$, three covariates PLTLTS, AGE and SEX are included:
	\begin{align*}
		h(t) = h_0(t)\cdot \exp\Big(\beta_1\cdot \text{PLTLTS} + \beta_2\cdot \text{AGE} + \beta_3\cdot \text{SEX}\Big)
	\end{align*}
	For model $3$, only two covariates PLTLTS and AGE are included:
	\begin{align*}
		h(t) = h_0(t)\cdot \exp\Big(\beta_1\cdot\text{PLTLTS} + \beta_2\cdot\text{AGE}\Big)
	\end{align*}
	For model $4$, only two covariates PLTLTS and SEX are included:
	\begin{align*}
		h(t) = h_0(t)\cdot \exp\Big(\beta_1\cdot\text{PLTLTS} + \beta_2\cdot\text{SEX}\Big)
	\end{align*}
	For model $5$, only one covariate PLTLTS is included:
	\begin{align*}
		h(t) = h_0(t)\cdot\exp\Big(\beta_1\cdot \text{PLTLTS}\Big)
	\end{align*}
	For part $(B)$:\vskip 2mm
	To see the hazard ratio for the effect of platelet variable (adjusted for age and sex):\vskip 2mm
	For model $1$:\vskip 2mm
	If SEX$= 1$(male), then we have:
	\begin{align*}
		\text{HR} &= \frac{h_0(t)\cdot \exp\Big(\beta_1\cdot 1 + \beta_2\cdot \text{AGE} + \beta_3\cdot 1 + \beta_4\cdot 1\cdot \text{AGE} + \beta_5\cdot 1 \cdot 1\Big)}{h_0(t)\cdot \exp\Big(\beta_1\cdot 0 + \beta_2\cdot \text{AGE} + \beta_3\cdot 1 + \beta_4\cdot 0\cdot \text{AGE} + \beta_5\cdot 0 \cdot 1\Big)}\\
		&= \exp\Big(\beta_1 + \beta_4\cdot \text{AGE} + \beta_5\Big)
	\end{align*}
	If SEX$= 0$(female), then we have:
	\begin{align*}
		\text{HR} &= \frac{h_0(t)\cdot \exp\Big(\beta_1\cdot 1 + \beta_2\cdot\text{AGE} + \beta_3\cdot 0 + \beta_4\cdot 1\cdot \text{AGE} + \beta_5\cdot 1 \cdot 0\Big)}{h_0(t)\cdot\exp\Big(\beta_1\cdot 0 + \beta_2\cdot \text{AGE} + \beta_3\cdot 0 + \beta_4\cdot 0\cdot \text{AGE} + \beta_5\cdot 0 \cdot 0\Big)}\\
		&= \exp\Big(\beta_1+ \beta_4\cdot \text{AGE}\Big)
	\end{align*}
	For model $2$:\vskip 2mm
	since there is no interaction, we simply have:
	\begin{align*}
		\text{HR} &= \frac{h_0(t)\cdot \exp\Big(\beta_1\cdot 1 + \beta_2\cdot \text{AGE} + \beta_3\cdot \text{SEX}\Big)}{h_0(t)\cdot\exp\Big(\beta_1\cdot 0 + \beta_2\cdot \text{AGE} + \beta_3\cdot \text{SEX}\Big)}\\
		&= \exp\Big(\beta_1\Big)
	\end{align*}
	For model $3$, $4$ and $5$:\vskip 2mm
	since there is no interaction, we would get the same result for hazard ratio on effect of PLTLTS, which is:
	\begin{align*}
		\text{HR} &= \exp\Big(\beta_1\Big)
	\end{align*}
	For part $(C)$:\vskip 2mm
	Adjusting for $\text{AGE} = 40$ and $\text{SEX} = 1$, the hazard ratios are:\vskip 2mm
	For model $1$:
	\begin{align*}
		\hat{\text{HR}} &= \exp\Big(\hat{\beta}_1 + \hat{\beta}_4\cdot \text{AGE} + \hat{\beta}_5\Big)\\
		&= \exp\Big(\hat{\beta}_1 + 40\hat{\beta}_4 + \hat{\beta}_5\Big)\\
		&= \exp\Big(0.470 + 40\times (-0.008) -0.503\Big)\\
		&= 0.7026
	\end{align*}
	For model $2$:
	\begin{align*}
		\hat{\text{HR}} &= \exp\Big(\hat{\beta}_1\Big) = \exp\Big(-0.725\Big) = 0.4843
	\end{align*}
	For model $3$:
	\begin{align*}
		\hat{\text{HR}} &= \exp\Big(\hat{\beta}_1\Big) = \exp\Big(-0.706\Big) =0.4936
	\end{align*}
	For model $4$:
	\begin{align*}
		\hat{\text{HR}} &= \exp\Big(\hat{\beta}_1\Big) = \exp\Big(-0.705\Big) = 0.4941
	\end{align*}
	For model $5$:
	\begin{align*}
		\hat{\text{HR}} &= \exp\Big(\hat{\beta}_1\Big) = \exp\Big(-0.694\Big) = 0.4996
	\end{align*}
	For part $D$:\vskip 2mm
	According to the output of model $1$, the effect of both interactions are not significant since one is with p value $0.850$ and the other is with p value $0.532$.\vskip 2mm
	For part $E$:\vskip 2mm
	according to model $2$ to $5$, PLTLTS is only marginally significant regardless controlling for SEX and(or) AGE or not, and SEX and AGE effect are not significant in these models, hence we do not need to control age and(or) sex as confounders.\vskip 2mm
	For part $F$:\vskip 2mm
	I would choose to report model $5$. First of all as we explained in part $E$ that the other covariates do not need to be controlled for potential confounders, also from part $D$ we see no interaction effects.\vskip 2mm
	For part $G$:\vskip 2mm
	Now we look at the output of model $5$, it reports that the effect of platelet on survival is marginally significant (with a p value of $0.08$).With a hazard ratio of $0.500$, the chance of death with normal platelets at diagnosis is half of those with abnormal platelets.\vskip 2mm
	Thus completed the answers for question $\#4$.
\end{sol}

Question $\# 5$.
\begin{sol}
	For part $(A)$:\vskip 2mm
	I hereby confirm that I have understood both ways using statement inside proc phreg and using counting process input while dealing with time-dependent covariates.\vskip 2mm
	For part $(B)$:\vskip 2mm
	For part $(i)$:\vskip 2mm
	we run the following code which is copied from page $165$:
	\begin{center}
		\includegraphics[width = 12cm]{Q5B01.jpg}
	\end{center}
	The output is:
	\begin{center}
		\includegraphics[width = 10cm]{Q5B02.jpg}
	\end{center}
	The financial aid is marginally significant (p value $0.057$), those who received financial aid has $0.694$ as much chance of being arrested as those who did not.\vskip 2mm
	Age is significant (p value $0.02$), and for every one year older in age, there is $95\%$ as much chance of being arrested as those who are one year younger.\vskip 2mm
	Race is not significant(p value $0.33$), but it does predict that those who are black has $1.35$ as much chance of getting arrested as those who are not black.\vskip 2mm
	Working experience is not significant (p value $0.81$).\vskip 2mm
	Marital status is not significant (p value $0.34$).\vskip 2mm
	Parole status is not significant (p value $0.88$).\vskip 2mm
	Number of convictions is significant (p value $0.0015$) and people with one more convinction will have about $9.5\%$ more chance of getting arrested.\vskip 2mm
	cumulative employemnt proportion is significant (p value $0027$).\vskip 2mm
	For part $(ii)$:\vskip 2mm
	The following SAS code handles the problem fully inside proc phreg, and it produce the same code as above:
	\begin{center}
		\includegraphics[width = 14cm]{Q5B03.jpg}
	\end{center}
	\begin{center}
		\includegraphics[width = 10cm]{Q5B04.jpg}
	\end{center}
	For part $(C)$:\vskip 2mm
	For part $(i)$, \vskip 2mm
	the following code using the number of switches as a time dependent covariate, adjusting for all the other covariates:
	\begin{center}
		\includegraphics[width = 14cm]{Q5C01.jpg}
	\end{center}
	The output is:
	\begin{center}
		\includegraphics[width = 10cm]{Q5C02.jpg}
	\end{center}
	we see that the number of switches in employement status is highly significant, and every one more switch will make it almost twice as possible(hazard ratio $1.95$) of getting arrested.\vskip 2mm
	For part $(ii)$:\vskip 2mm
	The following code using the number of negative switch as a time dependent covariate, adjusting for all the other covariates:
	\begin{center}
		\includegraphics[width = 14cm]{Q5C03.jpg}
	\end{center}
	The output is:
	\begin{center}
		\includegraphics[width = 10cm]{Q5C04.jpg}
	\end{center}
	we see that the number of negative switch is significant(p value $0.009$), and every unit increase of number of negative switch will make it $25\%$ less possible to get arrested.\vskip 2mm
	we are not surprised to see this happening, in fact it is possible of having either increased or decreased chance of getting arrested when the number of negative switch increases. Because in order to have more negative switch, you also need to have more positive switch. You could not unhire a person who has alrady lost his job.\vskip 2mm
	This completed the answers for question $\#5$.
\end{sol}

Question $\#6$.
\begin{sol}
	We add two dummy covariate here for our case. 
	\begin{align*}
		\text{switchtype}\_1 &= \left\{\begin{array}{ll} 1& \text{ if }\# \text{ of job lost }= 0\\ 0& \text{otherwise} \end{array}\right.\\
		\text{switchtype}\_2 &= \left\{\begin{array}{ll} 1& \text{ if }\# \text{ of job lost }= 1\\ 0& \text{otherwise} \end{array}\right.
	\end{align*}
	Of course they are time dependent as the number of weeks progress from week $1$ to week $52$.\vskip 2mm
	The following code fit these two time dependent covariates along with the others into the recidivism data:
	\begin{center}
		\includegraphics[width = 16cm]{Q601.jpg}
	\end{center}
	The output is:
	\begin{center}
		\includegraphics[width = 10cm]{Q602.jpg}
	\end{center}
	From the output we see that $\text{switchtype}\_1$ is significant(p value $0.01$). It tells that there is a significant difference of arresting time between people who have never lost the job and the people who have lost the job for at least two times. Those who have never lost a job before actually are almost twice as much possible of getting arrested as those who have lost job at least twice(hazard ratio $1.923$). This may sound ridiculous, however if we think about it, those who have never lost job before may have never been hired at all. Those who have lost job at lest twice must have been hired at least twice as well. Wile they are working the chance of getting arrested should actually be smaller.\vskip 2mm
	On the other hand, $\text{switchtype}\_2$ is not significant(p value $0.24$), and we do not have enough evidence showing that there is a different arresting time between those who have lost job once and those who have lost job at least twice.
\end{sol}





























\end{document}
